% !TEX TS-program = xelatex+makeindex+bibtex+shellescape
% !TEX encoding = UTF-8 Unicode

% Copyright 2024 Advaith Menon/GaTech

% Permission is hereby granted, free of charge, to any person obtaining
% a copy of this software and associated documentation files (the 
% "Software"), to deal in the Software without restriction, including
% without limitation the rights to use, copy, modify, merge, publish,
% distribute, sublicense, and/or sell copies of the Software, and to
% permit persons to whom the Software is furnished to do so, subject
% to the following conditions:

% The above copyright notice and this permission notice shall be
% included in all copies or substantial portions of the Software.

% THE SOFTWARE IS PROVIDED “AS IS”, WITHOUT WARRANTY OF ANY KIND,
% EXPRESS OR IMPLIED, INCLUDING BUT NOT LIMITED TO THE WARRANTIES
% OF MERCHANTABILITY, FITNESS FOR A PARTICULAR PURPOSE AND
% NONINFRINGEMENT. IN NO EVENT SHALL THE AUTHORS OR COPYRIGHT
% HOLDERS BE LIABLE FOR ANY CLAIM, DAMAGES OR OTHER LIABILITY,
% WHETHER IN AN ACTION OF CONTRACT, TORT OR OTHERWISE, ARISING
% FROM, OUT OF OR IN CONNECTION WITH THE SOFTWARE OR THE USE OR
% OTHER DEALINGS IN THE SOFTWARE.

% Observations - Tracker
\subsection{Model (with drag)}
\begin{frame}{Model (with drag)}{Data needed for simulation}
\begin{itemize}
    \item Net acceleration on object, \texttt{a\_net} \( = -g\hat{j}\)
     ms\textsuperscript{-2} \( = -9.8\hat{j}\) ms\textsuperscript{-2}
    \item Vertical displacement, \texttt{delta\_h} \( = -1.17\hat{j}\) m
    \item Initial position \( = 1.12\times 10^{-2} \hat{j} \) m
    \item Mass of the ball, \(m_{ball} = 4.6\ \mathrm{g} = 4.6 \times 10^{-3}
    \ \mathrm{kg}\)
    \item \textbf{NOTE}: The data stays the same, the only thing to keep in 
    mind is that we have to account for drag force too!
    \item \texttt{drag\_force} = \(b|v|^2 \hat{j}\)
\end{itemize}
\end{frame}

\begin{frame}[fragile]{Model (with drag)}{Simulation code}
\begin{multicols}{2}
\begin{minted}[breaklines,fontsize=\tiny]{python}
# CONSTANTS
ACC_G = 9.8 # ms-2
B = .001 # guessed by trial-and-error

ball = sphere(color=color.blue, radius=0.22) # sphere ball = new sphere(color.BLUE, 0.22);
trail = curve(color=color.green, radius=0.02)
origin = sphere(pos=vector(0,0,0), color=color.yellow, radius=0.04)
plot = graph(title="Position vs Time", xtitle="Time (s)", ytitle="Position (m)")
poscurve = gcurve(color=color.green, width=4)
plot = graph(title="Velocity vs Time", xtitle="Time (s)", ytitle="Velocity (m/s)")
velcurve = gcurve(color=color.green, width=4)
plot = graph(title="Acceleration vs Time", xtitle="Time (s)", ytitle="Velocity (m/s2)")
acccurve = gcurve(color=color.green, width=4)

# SYSTEM PROPERTIES & INITIAL CONDITIONS
ball.m = 4.6e-3 #kg
ball.pos = vector(0,1.12e-2,0) # m
ball.vel = vector(0,0,0) # 
ball.acc = vector(0, -ACC_G, 0) #ms-2

# Time
t = 0           # where the clock starts
deltat = 0.001   # size of each timestep

while ball.pos.y >= -1.157057:
    rate(1000)

    # Apply the Momentum Principle (Newton's 2nd Law)
    acc = (ball.acc + (B*ball.vel.y**2/ball.m)*vector(0, 1, 0))
    ball.vel = ball.vel + acc*deltat
    ball.pos = ball.pos + ball.vel*deltat
    
    t = t + deltat
    trail.append(pos=ball.pos)

    poscurve.plot(t,ball.pos.y)
    velcurve.plot(t,ball.vel.y)
    acccurve.plot(t,acc.y);
    print(t,ball.pos.x)

print("All done!")
\end{minted}
\end{multicols}
\end{frame}