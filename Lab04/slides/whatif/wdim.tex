% !TEX TS-program = xelatex+makeindex+bibtex+shellescape
% !TEX encoding = UTF-8 Unicode

% Copyright 2024 Advaith Menon/GaTech

% Permission is hereby granted, free of charge, to any person obtaining
% a copy of this software and associated documentation files (the 
% "Software"), to deal in the Software without restriction, including
% without limitation the rights to use, copy, modify, merge, publish,
% distribute, sublicense, and/or sell copies of the Software, and to
% permit persons to whom the Software is furnished to do so, subject
% to the following conditions:

% The above copyright notice and this permission notice shall be
% included in all copies or substantial portions of the Software.

% THE SOFTWARE IS PROVIDED “AS IS”, WITHOUT WARRANTY OF ANY KIND,
% EXPRESS OR IMPLIED, INCLUDING BUT NOT LIMITED TO THE WARRANTIES
% OF MERCHANTABILITY, FITNESS FOR A PARTICULAR PURPOSE AND
% NONINFRINGEMENT. IN NO EVENT SHALL THE AUTHORS OR COPYRIGHT
% HOLDERS BE LIABLE FOR ANY CLAIM, DAMAGES OR OTHER LIABILITY,
% WHETHER IN AN ACTION OF CONTRACT, TORT OR OTHERWISE, ARISING
% FROM, OUT OF OR IN CONNECTION WITH THE SOFTWARE OR THE USE OR
% OTHER DEALINGS IN THE SOFTWARE.

% Conclusion

\subsection{What does it mean?}
\begin{frame}{What does it mean?}{Validity of Energy Principle}
\begin{itemize}
    \item \textbf{Q}: For your model system, is the energy principle satisfied? Justify your answer by discussing briefly your plots of energy changes.
    \item \(W_{ext} = 0 \implies \Delta E = 0\), hence valid
\end{itemize}
\end{frame}

\begin{frame}{What does it mean?}{Oscillation Period}
\begin{itemize}
    \item \textbf{Q}: Using the data you obtained in Tracker, make two separate estimates of oscillation periods: first, by estimating the period of oscillation from the x position data and second, by estimating the period of oscillation from the y position data. Compare the two estimates and discuss.
    \item \textit{Time period is the difference between two successive crests/trough in the x-t graph} - stems from the definition ``the time taken for one complete to-and-fro oscillation.
    \item \(T_x = 1.425\ \mathrm{s}\)
    \item \(T_y = 1.330\ \mathrm{s}\)
    \item Difference because there are two different SHM's in two directions - x and y.
    \item Other causes could be drag and tracker inaccuracy.
\end{itemize}
\end{frame}