% !TEX TS-program = xelatex+makeindex+bibtex+shellescape
% !TEX encoding = UTF-8 Unicode

% Copyright 2024 Advaith Menon/GaTech

% Permission is hereby granted, free of charge, to any person obtaining
% a copy of this software and associated documentation files (the 
% "Software"), to deal in the Software without restriction, including
% without limitation the rights to use, copy, modify, merge, publish,
% distribute, sublicense, and/or sell copies of the Software, and to
% permit persons to whom the Software is furnished to do so, subject
% to the following conditions:

% The above copyright notice and this permission notice shall be
% included in all copies or substantial portions of the Software.

% THE SOFTWARE IS PROVIDED “AS IS”, WITHOUT WARRANTY OF ANY KIND,
% EXPRESS OR IMPLIED, INCLUDING BUT NOT LIMITED TO THE WARRANTIES
% OF MERCHANTABILITY, FITNESS FOR A PARTICULAR PURPOSE AND
% NONINFRINGEMENT. IN NO EVENT SHALL THE AUTHORS OR COPYRIGHT
% HOLDERS BE LIABLE FOR ANY CLAIM, DAMAGES OR OTHER LIABILITY,
% WHETHER IN AN ACTION OF CONTRACT, TORT OR OTHERWISE, ARISING
% FROM, OUT OF OR IN CONNECTION WITH THE SOFTWARE OR THE USE OR
% OTHER DEALINGS IN THE SOFTWARE.

% Observations - Tracker
\subsection{Code}

\begin{frame}[fragile]{Simulation code}{Finding mass of black hole}
\begin{multicols}{2}
\begin{minted}[breaklines,fontsize=\tiny]{python}
seed="amenon301@gatech.edu"

# ... kepler's formula

scene.background = color.white

star = sphere(color=color.yellow, radius=3e11)
trail = curve(color=color.green)
origin = sphere(pos=vector(0,0,0), color=color.black, radius=3e11)

arrowFnet = arrow(pos=star.pos, axis=vector(0,0,0), color=color.orange)
arrowFnet_par = arrow(pos=star.pos, axis=vector(0,0,0), color=color.magenta)
arrowFnet_perp = arrow(pos=star.pos, axis=vector(0,0,0), color=color.cyan)

G = 6.7e-11

# Mass of the star -- EDIT THIS LINE
star.m = 2e30

# Time interval between imported data points -- EDIT THIS LINE
deltat = 86400

# Time for first observation data point (default t=0)
t = 0

# Calculate velocity from data
Xvel = [0 for i in range(len(X)-1)]  
Yvel = [0 for i in range(len(X)-1)]  

idx = 0
while idx < (len(X)-1):
    # compute x component of velocity
    deltaX = X[idx+1] - X[idx]
    Xvel[idx] = deltaX/deltat
    # compute y component of velocity 
    deltaY = Y[idx+1] - Y[idx]
    Yvel[idx] = deltaY/deltat
    idx = idx + 1

t = t + deltat 

idx=1

myballs = 0

while idx <(len(X)-1):
    rate(50) 

    v_init = vector(Xvel[idx-1],Yvel[idx-1],0)
    v_final = vector(Xvel[idx],Yvel[idx],0)

    # EDIT THESE TWO LINES
    p_init = star.m * v_init 
    p_final = star.m * v_final # $\vec{p} = m\vec{v}$
    
    # EDIT THESE TWO LINES
    deltav = v_final - v_init
    deltap = p_final - p_init
    
    # EDIT THIS LINE
    dpdt = deltap/deltat

    # EDIT THIS LINE
    Fnet = dpdt

    # DIT THESE THREE LINES
    phat = hat((p_init + p_final) / 2) # in same dir as momentum 
    dmagpdt = (dot(Fnet, phat)) # which is |F|.1.cos (angle between phat and Fnet)
    dpdt_par = dmagpdt * phat # which is |F|.1.cos (angle between phat and Fnet).phat

    # EDIT THIS LINE
    Fnet_par = dpdt_par # yeah

    # EDIT THIS LINE
    dpdt_perp = Fnet - Fnet_par # components

    
\end{minted}
\end{multicols}
\end{frame}

\begin{frame}[fragile]{Simulation code}{Finding mass of black hole}
\begin{multicols}{2}
\begin{minted}[breaklines,fontsize=\tiny]{python}
# EDIT THIS LINE
    Fnet_perp = dpdt_perp # yes.

    # Calculate the mass of the black hole 
    # EDIT AND ADD LINES AS NEEDED
    #
    # or, F = (G*star.m*mBH)/(r^2)
    # mBH = (F * r^2)/(G * star.m)
    
    mBH = (mag(Fnet) * (X[idx]**2 + Y[idx]**2))/(G * star.m)
    myballs += mBH
    print("Mass of Black Hole", mBH)
    # print("Velocity of ball", v_init)

    # Update current position and velocity and show the object's current track
    star.pos = vector(X[idx],Y[idx],0)
    trail.append(pos=star.pos)

    # EDIT THE NEXT SEVEN LINES
    arrowscale = 1E-17     # determines how long to draw the arrows that represent vectors
    arrowFnet.pos = star.pos
    arrowFnet.axis = Fnet*arrowscale
    arrowFnet_par.pos = star.pos
    arrowFnet_par.axis = Fnet_par*arrowscale
    arrowFnet_perp.pos = star.pos
    arrowFnet_perp.axis = Fnet_perp*arrowscale
    
    t = t + deltat
    idx = idx + 1

print("Average mass of black hole", myballs/(idx-1))
print("|Fnet|=",mag(Fnet))
print("|Fnet_par|=",mag(Fnet_par))
print("|Fnet_perp|=",mag(Fnet_perp))
print("All done!")
    
\end{minted}
\end{multicols}
\end{frame}

\begin{frame}[fragile]{Simulation code}{Finding trajectory of the ship}
\begin{multicols}{2}
\begin{minted}[breaklines,fontsize=\tiny]{python}
Glowscript 3.2 VPython
## PHYS 2211
## Lab 3: Black Hole - Part 2
## 2211-lab3workingPART2.py
## Last updated: 2022-05-14 EAM


        
        

## =====================================
## VISUALIZATION & GRAPH INITIALIZATION
## =====================================

# White background so the black hole is visible
scene.background = color.white

# Visualization (object, trail, origin)
bh = sphere(pos=vec(0,0,0), color=color.black, radius=2e8)
planet = sphere(pos=vec(1e10,0,0), color=color.cyan, radius=1e8)
trailplanet = curve(color=color.cyan)
ship = sphere(pos=vec(planet.pos.x,planet.radius,0), color=color.magenta, radius=3e7)
trailship = curve(color=color.magenta)


    
scene.autoscale = False


## =======================================
## SYSTEM PROPERTIES & INITIAL CONDITIONS 
## =======================================


# Mass of the black hole 
# EDIT THIS LINE with your result from Part 1 of the lab
bh.m = 1.35641e+34


# Other constants - DO NOT CHANGE THESE FIVE LINES
G = 6.7e-11             # gravitational constant
planet.m = 9e28         # mass of planet
ship.m = 7e10           # mass of spaceship
deltat = 1              # deltat (in seconds)
t=0                     # initial time
# DO NOT CHANGE THE ABOVE FIVE LINES

def f_g(m_1, m_2, r, debug=False):
    rmag = r
    if isinstance(r, vec):
        rmag = mag(r)
    mag_fg = (G * m_1 * m_2) /( rmag**2 )
    if debug: print(hat(r))
    if isinstance(r, vec):
        return mag_fg * hat(r)
    return mag_fg

\end{minted}
\end{multicols}
\end{frame}

\begin{frame}[fragile]{Simulation code}{Finding trajectory of the ship}
\begin{multicols}{2}
\begin{minted}[breaklines,fontsize=\tiny]{python}
# Maximum time to run the simulation
# Default is ONE HOUR, but you can change it to a longer
# amount (e.g., 2hrs, 5hrs, etc) if you need more time to
# visualize a full orbit
tmax=60*60*3  


# Initial conditions for the planet 
#planet_speed = sqrt(G * bh.m / mag(planet.pos) )
#planet.vel = vec(0,planet_speed,0)
#print("Planet's orbital speed:", planet_speed, "m/s")
#print("Planet's orbital period:", (2*pi*mag(planet.pos)/planet_speed)/(60*60), "hrs")
#print("---")
#
#
## Initial conditions for the spaceship
## EDIT THESE LINES as needed to change the initial velocity of the ship
## Remember that your spaceship cannot move faster than the speed of light
#ship_speed_x = 0
#ship_speed_y = 0
#ship.vel = vec(ship_speed_x, ship_speed_y, 0)
##ship.vel = sqrt(2 * G * planet.m / planet.radius) * hat(planet.pos)
##print(ship.vel)

def resetConds():
    bh.pos = vec(0,0,0)
    planet.pos = vec(1e10,0,0)
    trailplanet.clear()
    ship.pos = vec(planet.pos.x,planet.radius,0)
    trailship.clear()
    planet_speed = sqrt(G * bh.m / mag(planet.pos) )
    planet.vel = vec(0,planet_speed,0)
    print("Planet's orbital speed:", planet_speed, "m/s")
    print("Planet's orbital period:", (2*pi*mag(planet.pos)/planet_speed)/(60*60), "hrs")
    print("---")
    
    
    # Initial conditions for the spaceship
    # EDIT THESE LINES as needed to change the initial velocity of the ship
    # Remember that your spaceship cannot move faster than the speed of light
#    ship_speed_x = 0
#    ship_speed_y = 0
#    ship.vel = vec(ship_speed_x, ship_speed_y, 0)
    #ship.vel = sqrt(2 * G * planet.m / planet.radius) * hat(planet.pos)
    #print(ship.vel)

\end{minted}
\end{multicols}
\end{frame}

\begin{frame}[fragile]{Simulation code}{Finding trajectory of the ship}
\begin{multicols}{2}
\begin{minted}[breaklines,fontsize=\tiny]{python}
# Maximum time to run the simulation
# Default is ONE HOUR, but you can change it to a longer
# amount (e.g., 2hrs, 5hrs, etc) if you need more time to
# visualize a full orbit
tmax=60*60*3  


# Initial conditions for the planet 
#planet_speed = sqrt(G * bh.m / mag(planet.pos) )
#planet.vel = vec(0,planet_speed,0)
#print("Planet's orbital speed:", planet_speed, "m/s")
#print("Planet's orbital period:", (2*pi*mag(planet.pos)/planet_speed)/(60*60), "hrs")
#print("---")
#
#
## Initial conditions for the spaceship
## EDIT THESE LINES as needed to change the initial velocity of the ship
## Remember that your spaceship cannot move faster than the speed of light
#ship_speed_x = 0
#ship_speed_y = 0
#ship.vel = vec(ship_speed_x, ship_speed_y, 0)
##ship.vel = sqrt(2 * G * planet.m / planet.radius) * hat(planet.pos)
##print(ship.vel)

def resetConds():
    bh.pos = vec(0,0,0)
    planet.pos = vec(1e10,0,0)
    trailplanet.clear()
    ship.pos = vec(planet.pos.x,planet.radius,0)
    trailship.clear()
    planet_speed = sqrt(G * bh.m / mag(planet.pos) )
    planet.vel = vec(0,planet_speed,0)
    print("Planet's orbital speed:", planet_speed, "m/s")
    print("Planet's orbital period:", (2*pi*mag(planet.pos)/planet_speed)/(60*60), "hrs")
    print("---")
    
    
    # Initial conditions for the spaceship
    # EDIT THESE LINES as needed to change the initial velocity of the ship
    # Remember that your spaceship cannot move faster than the speed of light
#    ship_speed_x = 0
#    ship_speed_y = 0
#    ship.vel = vec(ship_speed_x, ship_speed_y, 0)
    #ship.vel = sqrt(2 * G * planet.m / planet.radius) * hat(planet.pos)
    #print(ship.vel)

\end{minted}
\end{multicols}
\end{frame}

\begin{frame}[fragile]{Simulation code}{Finding trajectory of the ship}
\begin{multicols}{2}
\begin{minted}[breaklines,fontsize=\tiny]{python}
## =================
## CALCULATION LOOP
## =================
def calcLoop():
    t = 0
    while t < tmax:
        rate(1000)
    
        # Net force on the planet
        # EDIT THESE LINES (and/or add more if needed) to find:
        # 1) Fgrav on the planet by the black hole
        # 2) Fgrav on the planet by the spaceship
        # 3) the net force on the planet
        r_bh_to_planet = planet.pos
        F_bh_ON_planet = f_g(planet.m, bh.m, -r_bh_to_planet)
        
        r_ship_to_planet = planet.pos-ship.pos
        F_ship_ON_planet = f_g(ship.m, planet.m, -r_ship_to_planet)
        
        Fnet_ON_planet = F_bh_ON_planet + F_ship_ON_planet
        
        
        # Net force on the spaceship
        # EDIT THESE LINES (and/or add more if needed) to find:
        # 1) Fgrav on the spaceship by the black hole
        # 2) Fgrav on the spaceship by the planet
        # 3) the net force on the spaceship
        r_bh_to_ship = ship.pos
        F_bh_ON_ship = f_g(ship.m, bh.m, -r_bh_to_ship)
        
        r_planet_to_ship = ship.pos-planet.pos
        F_planet_ON_ship = f_g(ship.m, planet.m, -r_planet_to_ship)
    
        Fnet_ON_ship = F_bh_ON_ship + F_planet_ON_ship
        # print(Fnet_ON_ship)
    
    
        # MAKING THINGS MOVE   
        # Apply Newton's 2nd law and the position update formula 
        # to make the planet and spaceship move
        planet.vel = planet.vel + (Fnet_ON_planet/ planet.m) * deltat
        planet.pos = planet.pos + planet.vel * deltat
    
        ship.vel = ship.vel + (Fnet_ON_ship/ ship.m) * deltat
        # print(Fnet_ON_ship)
        ship.pos = ship.pos + ship.vel * deltat
\end{minted}
\end{multicols}
\end{frame}

\begin{frame}[fragile]{Simulation code}{Finding trajectory of the ship}
\begin{multicols}{2}
\begin{minted}[breaklines,fontsize=\tiny]{python}
        # Adding break conditions, in case of a crash
        # Do not edit these lines
        if mag(r_bh_to_ship) < bh.radius:
            # print("Oh no, you got sucked into the black hole!")
            # break
            return False
        if mag(r_ship_to_planet) < planet.radius:
            # print("Oh no, you crashed into the planet!")
            # break
            return False
    
        trailplanet.append(pos=planet.pos)      # Planet is cyan
        trailship.append(pos=ship.pos)          # Spaceship is magenta
        
        t = t+deltat
    return True;
    
def bruteForceIsHowNasaLaunchesSatellites():
    # for i in range(-int(3e8), int(3e8), int(1e5)):
        # for j in range(-int(3e7), int(3e7), int(1e6)):
            i = -1e8 # -1e1 # -1e8
            j = 1e7 # 1e7 # 1e7
            # reset conds
            resetConds()
            vel = vec(i, j, 0)
            ship.vel = vec(i, j, 0)
            print("Try! v =", vel)
            if calcLoop():
                print("This works! v =", vel)
        

bruteForceIsHowNasaLaunchesSatellites()

#print("---")
#print("Calculations finished after ",t/60,"minutes")
#print("The ship is now", mag(ship.pos)-bh.radius, "m away from the event horizon of the black hole")
#print("All done!")
\end{minted}
\end{multicols}
\end{frame}