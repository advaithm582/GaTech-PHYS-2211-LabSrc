% !TEX TS-program = xelatex+makeindex+bibtex+shellescape
% !TEX encoding = UTF-8 Unicode

% Copyright 2024 Advaith Menon/GaTech

% Permission is hereby granted, free of charge, to any person obtaining
% a copy of this software and associated documentation files (the 
% "Software"), to deal in the Software without restriction, including
% without limitation the rights to use, copy, modify, merge, publish,
% distribute, sublicense, and/or sell copies of the Software, and to
% permit persons to whom the Software is furnished to do so, subject
% to the following conditions:

% The above copyright notice and this permission notice shall be
% included in all copies or substantial portions of the Software.

% THE SOFTWARE IS PROVIDED “AS IS”, WITHOUT WARRANTY OF ANY KIND,
% EXPRESS OR IMPLIED, INCLUDING BUT NOT LIMITED TO THE WARRANTIES
% OF MERCHANTABILITY, FITNESS FOR A PARTICULAR PURPOSE AND
% NONINFRINGEMENT. IN NO EVENT SHALL THE AUTHORS OR COPYRIGHT
% HOLDERS BE LIABLE FOR ANY CLAIM, DAMAGES OR OTHER LIABILITY,
% WHETHER IN AN ACTION OF CONTRACT, TORT OR OTHERWISE, ARISING
% FROM, OUT OF OR IN CONNECTION WITH THE SOFTWARE OR THE USE OR
% OTHER DEALINGS IN THE SOFTWARE.

% Conclusion

\subsection{What does it mean?}
\begin{frame}{What does it mean?}{Parallel and perpendicular components}
\begin{itemize}
    \item From Newton's Second Law, we now that the net force is the change in net momentum per unit time.
    \item However, \(\frac{\Delta \vec{p}}{\Delta t} = \frac{\Delta m \times \Delta \vec{v}}{\Delta t}\), which means that we need mass, velocity and time taken to calculate rate of change of momentum.
    \item Force is an abstraction that can be used in other contexts, for example, the Work-Energy Theorem:
        \[\Delta K = W ( = \vec{F}\cdot \Delta s )\]
\end{itemize}
\end{frame}